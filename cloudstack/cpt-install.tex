\chapter{\cstack\ Installation}
\label{cpt:cloudstack}


\section{\acrlong{cms}}
\label{sct:cloudstack:cms}

It is recommended to install \cstack\ on CentOS~6.4 or higher.
During installation process we will need the following shell variables:
\begin{lstlisting}
MYSQLPASSWD
USERCLOUD
PASSWDCLOUD
\end{lstlisting}
Add the repository:
\begin{lstlisting}
cat >> /etc/yum.repos.d/cloudstack.repo << EOF
[cloudstack]
name=cloudstack
baseurl=http://cloudstack.apt-get.eu/rhel/4.2/
enabled=1
gpgcheck=0
EOF
\end{lstlisting}
Install packages:
\begin{lstlisting}
yum -y install ntp mysql-server cloudstack-management wget
\end{lstlisting}
Perform additional configuration:
\begin{lstlisting}
setenforce 0
sed s/"SELINUX=.*"/"SELINUX=permissive"/g -i /etc/selinux/config
ntpdate 0.centos.pool.ntp.org
service ntpd start
chkconfig ntpd on
mv /etc/my.cnf /etc/my.cnf.save
echo "
[mysqld]
datadir=/var/lib/mysql
socket=/var/lib/mysql/mysql.sock
user=mysql
# Disabling symbolic-links is recommended to prevent assorted security risks
symbolic-links=0

innodb_rollback_on_timeout=1
innodb_lock_wait_timeout=600
max_connections=700
log-bin=mysql-bin
binlog-format = 'ROW'
bind-address = 0.0.0.0
[mysqld_safe]
log-error=/var/log/mysqld.log
pid-file=/var/run/mysqld/mysqld.pid
" > /etc/my.cnf
service mysqld start
mysqladmin -u root password $MYSQLPASSWD
mysqladmin -u root -h $hostname  password $MYSQLPASSWD
chkconfig mysqld on
service mysql restart
\end{lstlisting}
CloudStack database setup:
\begin{lstlisting}
cloudstack-setup-databases $USERCLOUD:$PASSWDCLOUD@localhost \
  --deploy-as=root:$MYSQLPASSWD
\end{lstlisting}

If one is going to use HTTPS for connection to \acrshort{ec2api} and \acrshort{csapi}, it is required to use \cstack\ 4.4 or newer or to patch \cstack\ manually\footnote{\url{https://reviews.apache.org/r/17586}}.
Otherwise only HTTP configuration of \cstack\ will work correctly.
Configure \cstack\ for HTTPS:
\begin{lstlisting}
cloudstack-setup-management --https
\end{lstlisting}
Check configuration file /etc/cloudstack/management/ec2-service.properties:
\begin{lstlisting}
managementServer=127.0.0.1
cloudAPIPort=6443
cloudAPISSLEnabled=true
cloudstackVersion=2.2.0
WSDLVersion=2012-08-15
keystore=xes.keystore
keystorePass=apache
\end{lstlisting}
Option \texttt{cloudAPISSLEnabled=true} is available only for patched version of \cstack.
After this configuration is performed, it is highly recommended to change password (see Section~\ref{sct:config:passwd}). 
Also do not open ports to external network before changing password.


\section{\acrlong{ca}}
\label{sct:cloudstack:ca}

It is recommended to install \cstack\ on CentOS~6.4 or higher.
Add repository:
\begin{lstlisting}
cat  >>  /etc/yum.repos.d/cloudstack.repo << EOF
[cloudstack]
name=cloudstack
baseurl=http://cloudstack.apt-get.eu/rhel/4.2/
enabled=1
gpgcheck=0
EOF
\end{lstlisting}
Install packages:
\begin{lstlisting}
yum -y install ntp cloudstack-agent libvirt-client  qemu-kvm virt-manager
\end{lstlisting}
Perform additional configuration:
\begin{lstlisting}
setenforce 0
sed s/"SELINUX=.*"/"SELINUX=permissive"/g -i  /etc/selinux/config
ntpdate 0.centos.pool.ntp.org
service ntpd start
chkconfig ntpd on
sed s/"listen_tls.*"/"listen_tls = 0"/g -i /etc/libvirt/libvirtd.conf
sed s/"listen_tcp.*"/"listen_tcp = 1"/g -i /etc/libvirt/libvirtd.conf
sed s/"tcp_port.*"/"tcp_port = 16509"/g -i /etc/libvirt/libvirtd.conf
sed s/"auth_tcp.*"/"auth_tcp = \"none\""/g -i /etc/libvirt/libvirtd.conf
sed s/"mdns_adv.*"/"mdns_adv = 0"/g -i /etc/libvirt/libvirtd.conf
echo "LIBVIRTD_ARGS=--listen" >> /etc/sysconfig/libvirtd
service libvirtd restart
\end{lstlisting}


\section{Storage Engine}
One must configure an NFS server accessible from \acrshort{nets} and provide to folders for \acrshort{pse} and \acrshort{sse}.
In /etc/sysconfig/nfs uncomment the following lines:
\begin{lstlisting}
LOCKD_TCPPORT=32803
LOCKD_UDPPORT=32769
MOUNTD_PORT=892
RQUOTAD_PORT=875
STATD_PORT=662
STATD_OUTGOING_PORT=2020
\end{lstlisting}

To perform the next step one need access to \acrshort{sse} from \acrshort{cms} (in the configuration under consideration this takes place, otherwise one needs to mount \acrshort{sse} share on \acrshort{cms} to some folder, for example \texttt{\$DIR}).  
\texttt{\$DIR} is the location of \acrshort{sse} folder on \acrshort{cms}.
Now one need to execute the following command (please consult the \cstack\ guide for proper template name\footnote{\url{http://docs.cloudstack.apache.org/projects/cloudstack-installation/en/latest/installation.html\#prepare-the-system-vm-template}})
\begin{lstlisting}
 /usr/share/cloudstack-common/scripts/storage/secondary/cloud-install-sys-tmplt -m $DIR \
   -u http://d21ifhcun6b1t2.cloudfront.net/templates/4.2/systemvmtemplate-2013-06-12-master-kvm.qcow2.bz2 \
   -h kvm -F
\end{lstlisting}
