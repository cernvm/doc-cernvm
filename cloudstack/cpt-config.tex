\chapter{CloudStack Configuration and First Launch}
\label{cpt:config}

\section{Changing the Root Password}
\label{sct:config:passwd}

Following CloudStack documentation:
\begin{quote}
	During installation and ongoing cloud administration, you will need to log in to the UI as the root administrator. 
	The root administrator account manages the CloudStack deployment, including physical infrastructure. 
	The root administrator can modify configuration settings to change basic functionality, create or delete user accounts, and take many actions that should be performed only by an authorized person. 
	When first installing CloudStack, be sure to change the default password to a new, unique value.
	\begin{enumerate}
		\item Open your favorite Web browser and go to this URL.
			Substitute the IP address of your own Management Server:\\
			\url{https://<management-server-ip-address>:6443/client}
		\item Log in to the UI using the current root user ID and password. 
			The default is admin, password.
		\item Click Accounts.
		\item Click the admin account name.
		\item Click View Users.
		\item Click the admin user name.
		\item Click the Change Password button. \includegraphics[height=2ex]{figures/passwd}
		\item Type the new password, and click OK.
	\end{enumerate}
\end{quote}


\section{Adding Users and Keys}
\label{sct:config:accounts}

At least two users are required for standard \cstack\ configuration: administrator (default name admin) and non-privileged user.
Moreover in order to use \acrshort{csapi} or \acrshort{ec2api} one needs authorization keys. 
At first step we will create administrators keys following \cstack\ documentation:
\begin{quote}
	\begin{enumerate}
		\item Open your favorite Web browser and go to this URL.
			Substitute the IP address of your own Management Server:\\
			\url{https://<management-server-ip-address>:6443/client}
		\item Log in to the UI using the current root user ID and password. 
		\item Click Accounts.
		\item Click the admin account name.
		\item Click View Users.
		\item Click ``Generate keys''
		\item Click ``Yes''
		\item Copy API Key and Secret Key for future.
	\end{enumerate}
\end{quote}

Now we are able to use \acrshort{csapi}\footnote{\url{http://cloudstack.apache.org/docs/api}}.
Now let us create non-privileged user \emph{cvmuser} with account type ``User''.
\begin{quote}
	\begin{enumerate}
		\item Open your favorite Web browser and go to this URL.
			Substitute the IP address of your own Management Server:\\
			\url{https://<management-server-ip-address>:6443/client}
		\item Log in to the UI using the current root user ID and password. 
		\item Click Accounts.
		\item Click ``add accounts''
		\item Fill all fields and set Account type to User.
		\item Create user keys in the same way as for admin user.
	\end{enumerate}
\end{quote}


\section{Setting Global Variables}
\label{sct:config:globalvars}

For our particular task (launching \cernvm\ based elastic clusters) we need to change a number of global variables:
\begin{quote}
	\begin{enumerate}
		\item Open your favorite Web browser and go to this URL.
			Substitute the IP address of your own Management Server:\\
			\url{https://<management-server-ip-address>:6443/client}
		\item Log in to the UI using the current root user ID and password. 
		\item Click Global Settings
		\item Use find field to locate variables.
	\end{enumerate}
\end{quote}
Variables to be tuned:
\begin{lstlisting}[deletekeywords={true,enable}]
host                         "real host IP"  # example 195.19.226.152
management.network.cidr      "real cidr"     # example 195.19.226.129/27
expunge.delay                60
expunge.interval             60
expunge.workers              1               # usually you don't need more than one worker
max.account.primary.storage  400             # quota of primary storage for user in Gb,
                                             # depends on size of your storage
max.project.primary.storage  400             # quota of primary storage for project in Gb,
                                             # depends on size of your storage
enable.ec2.api               true
\end{lstlisting}
Now one need to restart \acrshort{cms} service on \host{A}:
\begin{lstlisting}
service cloudstack-management restart
\end{lstlisting}


\section{Enabling \acrshort{ec2api}}
For correct operation of \cernvm\ and elastic clusters one needs \acrshort{ec2api} configured.
To configure \acrshort{ec2api} global variable, \texttt{enable.ec2.api} must be set to true (see Section~\ref{sct:config:globalvars}). 
API access keys (see Section~\ref{sct:config:accounts}), private key and self-signed X.509 certificate are also required.
\begin{lstlisting}
EC2_AKEY="..."
EC2_SKEY="..."
cloudstack-aws-api-register --apikey=$EC2_AKEY --secretkey=$EC2_SKEY \
  --cert=~/.globus/usercert.pem --url=https://alice22.spbu.ru:5443/awsapi/
\end{lstlisting}


\section{Zone Creation, Adding Storages and Hosts}

Using web-interface.
\begin{enumerate}
	\item Open your favorite Web browser and go to this URL.
		Substitute the IP address of your own Management Server:\\
		\url{https://<management-server-ip-address>:6443/client}
	\item Log in to the UI using the current root user ID and password. 
		The default is admin, password
	\item Click Infrastructure
	\item Click add Zone
	\item Step by step create zone
	\begin{enumerate}
		\item type BASIC
		\item Hypervisor -- KVM
		\item Pod network -- \acrfull{netm}
		\item Guest network -- \acrshort{netg}. 
			\emph{\acrshort{netg} and \acrshort{netm} must be different!}
		\item host add -- put root username and  password
		\item Secondary storage provider -- NFS
	\end{enumerate}
\end{enumerate}
